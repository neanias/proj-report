% vim: textwidth=99 spell spelllang=en_gb:
% chktex-file 36

\documentclass[nobib, a4paper, twoside, justified]{tufte-book}

\usepackage[utf8]{inputenc}
\usepackage[british]{babel}
\usepackage{csquotes}
\usepackage[style=verbose, autocite=footnote, backend=biber]{biblatex}

\title{Displaying\\Heraldic\\Blazons}
\author{William Mathewson}
\publisher{University of Edinburgh}
\date{January 2018}

\addbibresource{bibliography.bib}

%%%%
%%
%% LOCALS are in `tufte-book-local.tex'

\usepackage{graphicx} % allow embedded images
\setkeys{Gin}{width=\linewidth,totalheight=\textheight,keepaspectratio}
\graphicspath{{graphics/}} % set of paths to search for images
\DeclareGraphicsExtensions{.pdf,.png}

\usepackage{amsmath}  % extended mathematics
\usepackage{booktabs} % book-quality tables
\usepackage{units}    % non-stacked fractions and better unit spacing
\usepackage{multicol} % multiple column layout facilities
\usepackage{fancyvrb} % extended verbatim environments
\usepackage{amsmath}
\usepackage{xspace}
\usepackage{makeidx}
\usepackage{mathtools}

\usepackage{float}
\usepackage{subfig}

\usepackage[final]{pdfpages}

%% Tell hyperref package it may break long URLs however it feels
\usepackage{hyperref}
\def\UrlBreaks{\do\/\do-}

%\newcommand{\dcim}{\emph{DCIM}\@\xspace}
%\newcommand{\ipam}{\emph{IPAM}\@\xspace}
%\newcommand{\dcimipam}{\emph{DCIM \& IPAM}\@\xspace}
%\newcommand{\desired}{\emph{desired}\@\xspace}
%\newcommand{\operational}{\emph{operational}\@\xspace}
%\newcommand{\wikid}{\textsc{WikiD}\@\xspace}
%\newcommand{\pywikid}{\textit{pywikid}\@\xspace}
\newcommand{\code}[1]{\texttt{#1}}
%\newcommand{\pythonthree}{Python$3$\@\xspace}

\newcommand{\svg}{\gls{svg}\@\xspace}
\newcommand{\svgs}{\glspl{svg}\@\xspace}
\newcommand{\dom}{\gls{dom}\@\xspace}

\newcommand{\charge}{\gls{charge}\@\xspace}
\newcommand{\charges}{\glspl{charge}\@\xspace}
\newcommand{\tincture}{\gls{tincture}\@\xspace}
\newcommand{\tinctures}{\glspl{tincture}\@\xspace}
\newcommand{\quarter}{\gls{quarter}\@\xspace}
\newcommand{\quarters}{\glspl{quarter}\@\xspace}

\newcommand{\blazon}{\gls{blazon}\@\xspace}
\newcommand{\blazons}{\glspl{blazon}\@\xspace}
\newcommand{\ublazon}{\Gls{blazon}\@\xspace}
\newcommand{\ublazons}{\Glspl{blazon}\@\xspace}

\newcommand{\payload}{\gls{payload}\@\xspace}
\newcommand{\payloads}{\glspl{payload}\@\xspace}

%\newcommand{\defeq}{\stackrel{\mathclap{\normalfont\mbox{def}}}{=}}
%\newcommand{\namedmap}[1]{\stackrel{\mathclap{\mbox{\small#1}}}{\longmapsto}}

%% R_{x->y} style
%\newcommand{\relate}[3]{\text{\itshape#1 }_{#2\ \mapsto\ #3}}

%% x --R--> y style
%\newcommand{\relate}[3]{#2\namedmap{#1}#3}

%% xRy style
%\newcommand{\relate}[3]{_{#2}#1_{#3}}

%\newcommand{\citetodo}{\sidenote{\todo{[Citation Needed]}}\@\xspace}
%\newcommand{\reftodo}{\sidenote{\todo{[Reference Needed]}}\@\xspace}
%\newcommand{\prooftodo}{\sidenote{\todo{[Substantiation Needed]}}\@\xspace}
\newcommand{\todo}[1]{{\noindent\textcolor{Red}{\textit{\quad#1}}\par}}

\let\lines\baselineskip{}

\usepackage[numberedsection=nameref]{glossaries}
\setacronymstyle{long-short}

\loadglsentries{glossary}

\glstoctrue{}

\begin{document}

\frontmatter

\maketitlepage{}

%\abstract{This is an example of {\tt infthesis} style. The file {\tt
%skeleton.tex} generates this document and can be used to get a ``skeleton''
%for your thesis. The abstract should summarise your report and fit in the
%space on the first page.
%%
%You may, of course, use any other software to write your report, as long as
%you follow the same style. That means: producing a title page as given here,
%and including a table of contents and bibliography.  }

\begin{publicationmeta}
  \section*{Acknowledgements}
  I would like to thank my supervisor, Julian Bradfield, for his help and advice as I was writing
  this project. I would also like to thank all my friends on Level 9 of Appleton Tower,
  particularly Connie, Nicole, Paul and Simon, who helped keep me going through the long, arduous
  journey that was this project and 4\textsuperscript{th} year as a whole.

  \section*{Declaration}
  I declare that this thesis was composed by myself,
  that the work contained herein is my own
  except where explicitly stated otherwise in the text,
  and that this work has not been submitted for any other degree or
  professional qualification except as specified.\par
  ({\textit{\thanklessauthor}})
\end{publicationmeta}

\tableofcontents

%\pagenumbering{Arabic}

\mainmatter%

\chapter{Introduction}%
\label{cha:introduction}

\section{Motivation}\label{sec:motivation}

In 1874 --- 4 years after his death --- John Papworth's \textit{Ordinary of British Armorials} was
published~\autocite{collins_1942}. In this work, he recorded approximately 50,000 descriptions of
families' coats of arms, none annotated.

This project makes it possible to have these descriptions, or \textit{\blazons} as they are termed
in heraldry (see~\ref{sec:heraldry}), drawn freely for people to view. This has potential
application for ancestry companies. Given the \blazon, they would be able to construct the shields
visually.

\section{Project Aim}%
\label{sec:project_aim}

This project was written as a paired project and as such, I only wrote the side of the project that
deals with rendering the \blazons, rather than the parsing. The parsing was implemented by my
partner, Anthony Gallagher, and, as such, this will not be covered in this report. Important shared
design elements will, however, be covered in Section~\ref{sec:core_concepts}.

The aim for this project was to designing and implementing a web app to render the parsed \blazon
by drawing \charges and \glspl{ordinary}.

\section{Contributions}%
\label{sec:contributions}

In this honours project, my contributions included:

\begin{itemize}
  \item Drawing the \charges and \quarters used on the shield, or \textit{\gls{escutcheon}},
  \item Writing the base web server and
  \item Writing the \quarter and \charge drawing algorithm.
\end{itemize}

\section{Report Structure}%
\label{sec:report_structure}

Following this chapter, the report is broken up into further 5 chapters with the following content:

\begin{itemize}
  \item \textbf{Chapter 2} presents the background surrounding the problem, with a beginner's
    guide to basic heraldry terms and the history surrounding heraldry, as well as other works
    related to this project;
  \item \textbf{Chapter 3} describes the overarching design decisions taken to guide the project in
    its implementation, along with its dependencies;
  \item \textbf{Chapter 4} provides an in-depth discussion of the iterative design and implementation
    process as I worked on solving the problem presented;
  \item \textbf{Chapter 5} evaluates the results, with regard to automated testing, other tools
    solving a similar, if not the same problem, and shortcomings that the final implementation
    had, and
  \item \textbf{Chapter 6} concludes the work with further work that could be undertaken.
\end{itemize}

\chapter{Background}\label{cha:background}

\section{Heraldry}\label{sec:heraldry}

Many families, countries and organisations --- primarily in Europe --- have coats of arms. Coats of
arms were initially used on shields on the battlefield to identify individual knights, but later
came to be used as flags and banners for individuals and families of the upper class at court. The
Royal Coat of Arms of the United Kingdom, belonging to the British monarch, can be seen in
Figure~\ref{fig:royal_coa}. If the reader wishes to learn more about heraldry and its history, I
would recommend the many heraldic works of Charles Boutell and John Brooke-Little.

\begin{marginfigure}
  \centering
  \def\svgwidth{0.8\linewidth}
  \input{graphics/Royal_Coat_of_Arms_of_the_United_Kingdom.pdf_tex}
  \caption{The Royal Coat of Arms of the United Kingdom.
  Source:~\url{https://upload.wikimedia.org/wikipedia/commons/9/98/Royal_Coat_of_Arms_of_the_United_Kingdom.svg}}%
  \label{fig:royal_coa}
\end{marginfigure}

At the centre of a coat of arms is a shield known as an \textit{\gls{escutcheon}}. The language
used to describe how the escutcheon is to be drawn is known as a \textit{\blazon}.
\ublazons have been used since the Norman conquest and have been refined to a regular language
in the process~\autocite{boutell_1864}, although, as John Brooke-Little said, ``many of the
supposedly hard and fast rules laid down in heraldic manuals [including those by heralds] are often
ignored.''~\autocite{brooke_little_1985} This blatant disregard for the rules introduces
difficulty in parsing the \blazons as the language loses some of its regularity.

\begin{marginfigure}
  \centering
  \def\svgwidth{0.8\linewidth}
  \input{graphics/Blason_Albert.pdf_tex}
  \caption{The shield of the town of Albert, France. \textit{Barry of ten argent and
  gules}. Source:~\url{https://upload.wikimedia.org/wikipedia/commons/e/ee/Blason_Albert.svg}}%
  \label{fig:blason_albert}
\end{marginfigure}

\ublazons have a few key attributes:
\begin{itemize}
  \item The \textit{\gls{field}}, which is the background colour of the shield or
    \quarter;
  \item \textit{\Glspl{ordinary}}, which are geometric shapes (as seen in Figure~\ref{fig:scrope},
    bearing a golden slash, or \textit{bend});
  \item \textit{\Glspl{charge}}, which are small emblems, such as fleur-de-lis and lions;
  \item \textit{\Glspl{variation}}, which describe how the field or \charge is patterned.
    \Glspl{variation} can indicate patterns such as chequered or coloured lines (as seen in
    Figure~\ref{fig:blason_albert}); and
  \item \textit{\Glspl{tincture}}, which are the colours and patterns for \charges, ordinaries
    and fields.
\end{itemize}

The tinctures are derived from Norman French and are divided into 3 groups, typically known as
\textit{metals}, \textit{colours} and \textit{furs}. In British heraldry, the colours are also
derived from Norman French and so the names appear archaic. In heraldry, blue is \textit{azure} and
red is \textit{gules} for instance. The metals are \textit{or} and \textit{argent}, for gold and
silver respectively. Whilst the tinctures are linked to colours, the College of Arms does not
specify which shade of that colour is required for the tinctures, leaving it to the artist to
decide~\autocite{college_of_arms_faq}. In this case, I have used default CSS colours, using the
English names, for instance `red' for \textit{gules}.

\begin{marginfigure}
  \centering
  \def\svgwidth{0.8\linewidth}
  \input{graphics/chief_example.pdf_tex}
  \caption{\textit{Purpure, a chief Gules}.}%
  \label{fig:chief_example}
\end{marginfigure}

\ublazons conventionally follow a form of starting with the \textit{\gls{tincture}} or
\textit{\gls{variation}} of the field. After the description of the field,
\textit{\glspl{ordinary}} and \textit{\charges} are named with their tinctures. An example of
this is ``\textit{Purpure, a chief Gules}''. This \blazon describes an escutcheon with a field of
\textit{purpure} (purple), with a \textit{Chief} \gls{ordinary} (a bar across the top of the
shield) of \textit{gules} (red). This can be seen (drawn by the project web app) in
Figure~\ref{fig:chief_example}.

\begin{marginfigure}
  \centering
  \def\svgwidth{0.8\linewidth}
  \input{graphics/scrope.pdf_tex}
  \caption{The Scrope \gls{escutcheon}; \textit{Azure, a bend Or}.}%
  \label{fig:scrope}
\end{marginfigure}

A simple, but notable, \blazon is that of the Scrope family. In the 14th century, the Baron Scrope
brought a case action against Sir Robert Grosvenor when he noticed that they both had the same coat
of arms. Many witnesses gave evidence in the case, including Geoffrey
Chaucer~\autocite{scrope_grosvenor}. The case was ultimately decided in Scrope's favour. The Scrope
coat of arms has a \blazon of \textit{Azure, a bend Or}; a depiction of this (as drawn by the
project web app written for this project) can be seen in Figure~\ref{fig:scrope}.

Whilst the Scrope arms are prominent in heraldry, they are simplistic and indicative of
medieval arms. Coats of arms became more complex as they developed through the centuries, with
instances of \textit{quarterly} shields, \textit{grand-quarterlies} --- quarterlies within
quarterlies --- and \textit{differenced} arms. \textit{Differenced} arms involve adding an
\gls{ordinary} over an existing coat of arms. This was typically used to differentiate similar
looking coats of arms, especially between father and sons. Common examples of differentiated
shields are seen in duchies' coats of arms, particularly those which were given to Charles II's
illegitimate children. Examples of more complex shields can be seen in
Figure~\ref{fig:complex_shields}.

\begin{figure*}[h]
  \subfloat[Neville, 16th Earl of Warwick's coat of arms. An example of grand-quarterlies and
  differenced arms. Source:~\url{https://upload.wikimedia.org/wikipedia/commons/d/d1/Neville_Warwick_Arms.svg}.]{%
    \def\svgwidth{0.3\linewidth} %
    \input{graphics/Neville_Warwick_Arms.pdf_tex}
  }
  \qquad
  \subfloat[A quarterly shield drawn by the web app. \textit{Quarterly: 1st and 4th: Gules, a bend
  Sable; 2nd and 3rd: Azure, a chief Or}. ]{%
    \def\svgwidth{0.28\linewidth}
    \input{graphics/quarterly_example.pdf_tex}
  }
  \caption{Some examples of more complex coats of arms.}\label{fig:complex_shields}
\end{figure*}

For a time, it was considered bad form to repeat a \textit{tincture} in a \blazon, and use a
reference to the tincture's previous use. The Heraldic Society gives an example as such:
``\textit{`Azure on a fess argent three billets azure'} [would have been written as] \textit{`Azure
on a fess argent three billets of the first'}''. The `\textit{of the first}' refers to the field's
tincture of azure. This \blazon describes a blue shield, with a white bar horizontally across the
middle with 3 white rectangles arranged along the bar. The Heraldic Society advocates repeating
tinctures to reduce ambiguity~\autocite{blazon_in_coa}.

\section{Related Works}%
\label{sec:related_works}

The problem at hand has been partially solved before: Robert Billard wrote
\textit{Blazons!}~\autocite{dos_blazon} for Windows 3.1 and 95 and Mark Shoulson and Arnt Richard
Johansen wrote \textit{pyBlazon}~\autocite{pyblazon} in 2008. \textit{Blazon!} does not parse a
blazon for automatic rendering, but provides an environment with predefined \charges and
\glspl{ordinary} to build one's own \glspl{escutcheon}. \textit{pyBlazon} solves the problem
exceptionally well, but only solves it for basic escutcheons, and does not render quarterly
shields. Thus, a primary goal of this project is to support this. \textit{pyBlazon}, as the name
suggests, is implemented in Python~\autocite{python} and hosted in a PHP webserver. My plan to implement the
rendering in TypeScript~\autocite{typescript} (see Section~\ref{sec:core_concepts}) would aid portability as the code
would be able to run on all modern browsers, whilst also being decoupled from the back-end,
allowing pluggable implementations of the parsing.

Despite these tools existing, many escutcheons drawn and uploaded to WikiMedia in
\svg~\autocite{ferraiolo2000scalable} format appear to have been created in
Inkscape~\autocite{inkscape}, rather than being created using the developed tools. Much work has
been done in collecting and cataloguing \blazons themselves (especially John Papworth as
mentioned in Section~\ref{sec:motivation}) giving plenty of examples against which to test the
finally completed application.

\section{Summary}%
\label{sec:background_summary}

In this chapter, we covered basic heraldry, including core terminology, as well as works related to
the project. Core heraldry terminology includes:

\pagebreak%

\begin{itemize}
  \item \textit{\Gls{escutcheon}} --- the shield in the coat of arms;
  \item \textit{\Gls{field}} --- the background of the escutcheon;
  \item \textit{\Glspl{ordinary}} --- geometric shapes on the escutcheon;
  \item \textit{\Glspl{charge}} --- small emblems, such as fleur-de-lis and lions; and
  \item \textit{\Gls{tincture}} --- the colours and patterns for \charges, ordinaries and fields.
\end{itemize}

All relevant heraldry terminology may be found in the Glossary on page~\pageref{main}.

\chapter{Design}%
\label{cha:design}

\section{Core Concepts}%
\label{sec:core_concepts}

The two languages chosen for implementing this project were Python~\autocite{python} and
TypeScript~\autocite{typescript}. Python seemed like an obvious choice with its good support for
\gls{nlp} through the \gls{nltk}~\autocite{bird2004nltk}. TypeScript is a typed superset of
JavaScript written by Microsoft that compiles to plain JavaScript.  TypeScript was chosen as a
nicer alternative to programming in pure JavaScript, thanks to the addition of powerful features
such as types, access control and abstract classes. As a note to the reader, to maintain
interoperability with JavaScript, TypeScript adds type definitions after the variable name, rather
than before it, as in C. An example being that an \texttt{int} defined in C would be \texttt{int
limit}, but in TypeScript, this would be \texttt{limit: number}. (JavaScript/TypeScript also has a
unified \texttt{number} type to handle both \texttt{float}s and \texttt{int}s.)

The core design for this project centres around having a split stack; with a Python back-end
parsing the \blazon, serialising it into JSON\footnote{\gls{json} is a lightweight data-interchange
  format, consisting of key-value pairs, array data types and other serialisable data types (such
  as strings, numbers and booleans). JSON is derived from JavaScript's associative array-style data
type, \texttt{Object}. An example of JSON can be seen in Figure~\ref{fig:expected_output}.} and
passing it to the TypeScript front-end, which would then draw it onto the webpage. This allowed for
large amounts of flexibility, enabling the two halves of the project to be developed in tandem with
the Separation of Concerns principle being adhered to throughout. It also allows for pluggable
rendering implementations as the JSON schema for drawing \payloads can be well-defined.

The Python back-end receives a JSON \payload from the webpage containing the \blazon; it parses the
\blazon using a Context-Free Grammar (CFG) parser and identifies the most important parts of the
\blazon. It then serialises these back into a JSON response to be sent to the webpage for rendering.
The specification was such that if the webpage was given a \blazon of ``Azure, a bend Or'', it would
return the JSON response seen in Figure~\ref{fig:expected_output}.

\begin{figure}[h]
  \begin{verbatim}
    {
      "field": "azure",
      "charges": [{
        "charge": "bend",
        "tincture": "or"
      }]
    }
  \end{verbatim}
  \caption{Expected response from the Python back end, for a given \blazon of ``Azure, a bend Or''.}%
  \label{fig:expected_output}
\end{figure}

The TypeScript front-end receives this response, applies the \texttt{azure} CSS class to the field
element, then draws a bend onto the field with an \texttt{or} CSS class. The rendering of this
response can be seen in Figure~\ref{fig:scrope}. The front-end does not use a framework and is just
written in plain TypeScript.

Escutcheons are drawn using the \svg format for portability across browsers as well as the
eponymous scalability of \svg images. This allows drawn escutcheons to be embedded elsewhere with
ease, either directly or through rendering the \svgs as other image formats via programs like
Inkscape.

For version control, \texttt{git}~\autocite{git} was used and all code was hosted on GitHub.

\section{External Dependencies}%
\label{sec:external_dependencies}

The front-end depends on a pair of libraries for \svg rendering and \dom manipulation: the
selection library of D3.js~\autocite{d3js} and jQuery~\autocite{jquery}. D3.js and jQuery both
provide many powerful functions for \svg and \dom manipulation. The Mozilla Developer Network
defines the \dom as such: ``The \glsentryfirst{dom} connects web pages to scripts or programming
languages. Usually that means JavaScript, but modelling HTML, SVG, or XML documents as objects is
not part of the JavaScript language. The \dom model represents a document with a logical tree.
Each branch of the tree ends in a node, and each node contains objects. \dom methods allow
programmatic access to the tree; with them you can change the document's structure, style or
content. Nodes can have event handlers attached to them. Once an event is triggered, the event
handlers get executed.''~\autocite{mdn_dom}.

For further assets, Sass~\autocite{sass-lang} (a CSS extension language) was used as a CSS
pre-processor and Bootswatch~\autocite{bootswatch-flatly} (a predefined CSS theme) was used for the
base styling.  Webpack~\autocite{webpack} was used to compile TypeScript down to JavaScript ---
minifying and uglifying it in the process --- and to concatenate all source files and their
dependencies into one main JavaScript file. \textit{\Gls{minification}} of JavaScript assets
involves stripping out all unnecessary whitespace and tokens.  \textit{\Gls{uglification}}
transforms the JavaScript code by renaming all variables and functions into short, obfuscated names
to reduce the footprint of the assets. These two techniques can decrease loading times of web apps
as the browser has to download smaller asset resources than the original, raw source code.

\subsection{Development Dependencies}%
\label{sub:development_dependencies}

To maintain code quality, TSLint~\autocite{tslint} (a TypeScript code linter) was used and set up to
automatically run as part of the Travis \gls{ci}~\autocite{travis} service, causing a build to fail
if the linter detected a style violation. For unit testing, Jest~\autocite{jest} (with
ts-jest~\autocite{ts-jest} for TypeScript support) was used, especially for its powerful mocking
and expectation matcher functionality. Automated documentation generation was provided by
TypeDoc~\autocite{typedoc}.

\section{Iterative Design}%
\label{sec:iterative_design}

An iterative design was used for writing the code for this project. Iterative design is a cyclical
process of designing, prototyping and evaluating. One designs and prototypes a new feature before
evaluating the final feature design. If the design is acceptable, the new feature is implemented,
otherwise the cycle restarts. An iterative design process enables one to address different functionality
as separate tasks, building on top of one another. It also allows for heavy refactors of a project
whilst staying in one cycle.

\chapter{Implementation}%
\label{cha:implementation}

\section{Basic Charge Rendering}%
\label{sub:basic_charge_rendering}

\subsection{First Design Iteration}%
\label{sub:first_design_iteration}

The first design iteration had a specific focus on basic \charge drawing, with a plan to begin a
second iteration for adding quarterly rendering with lessons learnt from this iteration.

I decided to have all drawing logic defined in the client, existing in a single module with minimum
dependencies. The shield outline was rendered on the page on load as part of the HTML template.
This helped support a reliable entry point for the drawing logic as it was able to easily select
the shield element to begin appending other \svg elements to. Appending these \svg elements
allowed layering to be achieved as \svg orders layers based on the order of elements in the
document. This was used to great effect when drawing quarters (see
Section~\ref{sub:second_design_iteration}).

The front-end was designed around functional paradigms; breaking up major functionality into
functions that would deal with smaller, encapsulated functionality, such as adding extra layers to
the HTML template or clearing the shield when drawing a new \blazon. This allowed for a stable API,
as the single point of access function would not be renamed but all other functions may be changed
and updated separately. As described in Section~\ref{sec:core_concepts}, the front-end first
accessed the \texttt{field} value in the parsed JSON response, applied the value as the CSS class
for the shield and then moved onto the \charges. It iterated over the \texttt{\charges} array in
the JSON response, drawing each \charge onto the shield and applying the tincture as the CSS class.
Due to \svg layering, as mentioned earlier, if there were multiple \charges specified in the
response, all would be drawn according to the array ordering.

\subsection{First Design Implementation}%
\label{sub:first_design_implementation}

As described in Section~\ref{sub:first_design_iteration}, the initial approach was to have all methods in
the core \texttt{index.ts} file that would be compiled and loaded in the browser. This
meant a smaller footprint when the code was bundled by Webpack and easier maintenance as all
relevant functions were next to one another, following the Step-Down
Rule~\autocite{martin2009clean}\footnote{The Step-Down Rule dictates that if function \texttt{A()}
calls function \texttt{B()} and \texttt{C()} in its function body, functions \texttt{B()} and
\texttt{C()} should be defined immediately after function \texttt{A()}.}.

\subsubsection{\texttt{drawShield(\blazon)}}%
\label{ssub:draw_shield}

The web app had a single entry point of \texttt{drawShield(\blazon)}, where \texttt{\blazon} was
the whole JSON \payload returned from the \texttt{/\_parse} endpoint (see
Figure~\ref{fig:expected_output} for an example JSON response). This presented a problem initially
as TypeScript didn't handle the unstructured parsed data well due to it being a JavaScript
\texttt{Object}\footnote{In JavaScript, an \texttt{Object} works as both an associative array and a
basis for classes and inheritance through its \texttt{prototype} field.} instance. Attempting to
access members of this object (such as \texttt{\gls{field}}) causes TypeScript to produce an error
that the contents might be undefined and thus return a \texttt{null} object. To fix this, I
designed \texttt{interface}s with the expected fields in the \payload; one for the whole object,
\texttt{IBlazon}, and one for the \charges array contained within, \texttt{ICharge}. Similarly, to
avoid problems with string matching, I defined 2 \texttt{enum}s to represent the supported
tinctures and \charges, \texttt{ETincture} and \texttt{ECharge} respectively. As discussed in
Section~\ref{sec:adding_quarterly_rendering}, I later added another \texttt{enum} for quarters.
All \texttt{interface}s and \texttt{enum}s can be found in Appendix~\ref{cha:interfaces_and_enums}.
Having fixed this data problem, \texttt{drawShield(\blazon)} was now able to access members of the
\blazon object safely.

\subsubsection{\texttt{clearShield()}}%
\label{ssub:clear_shield}

To avoid the problem of overlapping \charges, I had to write a \texttt{clearShield()} method that
would iterate over all the \texttt{path} nodes in the \svg document, and delete them. This,
however, promptly deleted the shield outline, so I had to add a check to prevent deleting
\texttt{path} nodes with a \texttt{\#shield} id, instead only removing the CSS class. Having
cleared the shield of any possible obstructions, the \texttt{drawShield} method would then assign
the contents of the \texttt{field} value as the CSS class and iterate over the \texttt{\charges}
array, passing each \texttt{\charge} to \texttt{drawCharge(charge: ICharge)}.

\subsubsection{\texttt{drawCharge()} and \texttt{ChargeShapes}}%
\label{ssub:draw_charge}

When each \charge node is created in \texttt{drawCharge}, it is assigned an id. This id is formed
from the name of the charge, followed by a random number in the range $1\text{--}512$ inclusive
with the hope that the range is large enough to lack overlaps. This was later changed to rely on a
cryptographically secure, random number generator.

To draw shapes in \svg a \texttt{path} node requires a `\texttt{d}' attribute which contains the
commands for drawing said shape. To generate all these attributes, I drew all the \charge shapes in
Inkscape and extracted the `\texttt{d}' attributes from the generated \svgs. At first, I put a
\texttt{Map}\footnote{A \texttt{Map} here is a TypeScript/JavaScript data type, also known as a
\texttt{HashMap} or an associative array.} of the \charges and their paths in the global scope,
available for all functions to access. This worked for \charges that were produced using a
\texttt{path} node with a `\texttt{d}' attribute, but introduced problems when using the
chief \charge (a chief \charge being displayed in Figure~\ref{fig:chief_example}). The
chief \charge was drawn using a \texttt{<rect>}\footnote{A \texttt{<rect>} node in \svg is used for
drawing rectangular shapes.} node which required `\texttt{x}' and `\texttt{y}'
co-ordinates to specify a starting point and \texttt{height} and \texttt{width} attributes to
describe the size of the rectangle. To address this, I wrote a \texttt{ChargeShapes} class to
encapsulate the \charges and their attributes. This class provided one public member,
\texttt{chargePaths} which was of the type \texttt{Map<string, Map<string, string>\hphantom>}.
Having this as a map allowed \texttt{drawCharge} to first check if the app knew how to draw the
\charge by checking whether \texttt{ChargeShapes.chargePaths} contained the \charge as a key.  If
the \charge had an entry, then \texttt{drawCharge} would iterate over the attribute \texttt{Map}
and apply them to the \texttt{path} or \texttt{rect} node, before finally applying the CSS class.

A visual representation of \texttt{drawCharge}'s execution may be seen in
Figure~\ref{fig:draw_charge}.

\begin{figure}
  \centering
  \subfloat[Initial \gls{escutcheon} outline on the webpage.]{%
    \def\svgwidth{0.45\linewidth}%
    \input{graphics/charge_renderer_process/draw_charge_stage0.pdf_tex}
  }
  \qquad
  \subfloat[First stage: applying the \texttt{field} CSS class.]{%
    \def\svgwidth{0.45\linewidth}%
    \input{graphics/charge_renderer_process/draw_charge_stage1.pdf_tex}
  }
  \\
  \subfloat[Second stage: drawing the charge onto shield.]{%
    \def\svgwidth{0.45\linewidth}%
    \input{graphics/charge_renderer_process/draw_charge_stage2.pdf_tex}
  }
  \qquad
  \subfloat[Final stage: applying the \texttt{tincture} CSS class on charge.]{%
    \def\svgwidth{0.45\linewidth}%
    \input{graphics/charge_renderer_process/draw_charge_stage3.pdf_tex}
  }
  \caption{The process of \texttt{drawCharge()} rendering a Scrope \gls{escutcheon}.}%
  \label{fig:draw_charge}
\end{figure}

The final part of \texttt{drawCharge} applied a transform to the \texttt{path} if the JSON response
included a boolean flag, \texttt{sinister}, to indicate that the bend \charge should be flipped.
This writes to an attribute \texttt{transform} which applies a \texttt{matrix} transformation that
flips the \charge followed by a \texttt{translate} transformation to move it into place. An
example of a sinister bend can be seen in Figure~\ref{fig:sinister_example}.

\begin{marginfigure}
  \centering
  \def\svgwidth{0.8\linewidth}
  \input{graphics/sinister_example.pdf_tex}
  \caption{\textit{Or, a bend sinister Vert.}}\label{fig:sinister_example}
\end{marginfigure}

\subsection{First Design Evaluation}%
\label{sub:first_design_evaluation}

Whilst this simple implementation worked well for drawing basic \glspl{escutcheon}, like
Figure~\ref{fig:chief_example} and Figure~\ref{fig:sinister_example}, it wasn't able to draw more
complex shields like those seen in Figure~\ref{fig:complex_shields} and thus was not meeting the
project aims.

\section{Adding Quarterly Rendering}%
\label{sec:adding_quarterly_rendering}

\subsection{Second Design Iteration}%
\label{sub:second_design_iteration}

\subsubsection{Design Principles}%
\label{ssub:design_principles}

Whilst prototyping for adding functionality to render quarterly shields, I found that it was going
to be impossible to maintain the initial, atomic design laid out in
Section~\ref{sub:first_design_iteration} whilst also keeping the code clean and readable. This started the
second design iteration of the project. In this iteration, I designed a new, modular system that
leaned more heavily into Object-Oriented paradigms than functional ones. This new design was
written to follow the principles of delegation, decoupling, the Single
Responsibility~\autocite{martin2000design} and Open/Closed principles~\autocite{martin2000design}.

The Single Responsibility principle dictates that a class should have one, and only one, reason to
change. In \textit{Design Principles and Design Patterns}~\autocite{martin2000design}, Martin
defines a responsibility as a reason to change, such that a class should only change if that one
responsibility changes. If another responsibility is introduced, it should be given its own class.
This principle works well with decoupling as responsibilities can be changed in individual classes
without affecting other classes that use it. The Open/Closed principle dictates that a class should
be open to extension but closed to modification. A typical example of this would be using abstract
classes; the interface specified by the abstract class is closed to modification, but the child
classes may extend the functionality in their implementation. Thus, another class depending on the
class extending the abstract class can rely on the interface without having to know about the
internals.

To comply with these principles, all major sections of functionality, including \blazon
\payload parsing, \charge and \quarter rendering, were encapsulated in their own classes with clear
names and well-defined, shared APIs.

\subsubsection{Top-Level Design}%
\label{ssub:top_level_design}

The new design had a top-level class of \texttt{\ublazon} which had a single public function,
\texttt{draw()}. The \texttt{draw} method, as before, would clear the shield and then delegate
drawing responsibility to specialised \texttt{\charge} and \texttt{\quarter} renderers by calling
their \texttt{draw()} methods. This \texttt{\ublazon} class became the new entry-point, being
instantiated and called in a \texttt{main()} function in \texttt{index.ts}. In a similar fashion to
the top-level \texttt{\ublazon} class, a \texttt{\Gls{quarter}} class was defined to enable proper
delegation for rendering both the \quarter \textit{and} the \charges contained within the quarter.
The new \texttt{ChargeRenderer} class would contain most major logic for drawing, as well as
\texttt{id} generation. \texttt{QuarterRenderer} extended \texttt{ChargeRenderer} to add
\quarter-specific logic, while also being able to call up to it to draw the contained \charges.

For a quarterly \blazon, the \payload would have the \texttt{\gls{field}} value set to
\texttt{``quarterly''} and rather then having a field \texttt{charges: ICharge[]}, it would contain
a field \texttt{quarters: IBlazon[]}. This worked as \quarters are treated as their own small
\glspl{escutcheon} in heraldry and are described as such. An example of a quarterly \blazon would
be \textit{Quarterly: 1st and 4th: Gules, a bend Sable; 2nd and 3rd: Azure, a chief Or}, as seen in
Figure~\ref{fig:complex_shields}. It was then possible to change functionality of the app depending
on the contents of the \texttt{field} key. To account for the new quarters that needed to be drawn,
a new \texttt{enum} \texttt{EQuarter} was designed, with the members being the positions of the
\svg elements and the \texttt{path}s in the \texttt{ChargeShapes.chargePaths} \texttt{Map}.

\subsubsection{Using \svgs for Great Good}%
\label{ssub:using_svg_for_great_good}

In this refactor, I also made use of more \svg properties: clip paths and the \texttt{<g>} element.
Clip paths allowed defining a path that cropped the element it was defined on. This was
particularly useful for drawing \charges inside of \quarters as the edges outside of the \quarter
would be cropped out by the clip path defined for that \quarter. The \texttt{<g>} element is a
grouping element for \svg; it applies all transformations defined on it to all its child elements
and any of its attributes are also inherited. I used the \texttt{<g>} elements for both grouping
together charges and quarters, but also explicitly named layers for the shield outline and the
\charges within.

\subsection{Second Design Implementation}%
\label{sub:second_design_implementation}

The Figure references in the following headings refer to the UML diagrams in
Appendix~\ref{sec:second_design_iteration_diagrams}.
%UML diagrams for all classes mentioned in this subsection can be found in
%Appendix~\ref{sec:second_design_iteration_diagrams}.

\subsubsection{\texttt{Blazon} (Figure~\ref{fig:BlazonUML})}%
\label{ssub:blazon}

\texttt{\ublazon} took 2 arguments in its \texttt{constructor}: \texttt{svg} and \texttt{data}.
\texttt{svg} was a D3.js~\autocite{d3js} data type that contained a selector for the \texttt{<svg>}
element in the HTML document, for appending elements to. This data type is henceforth referred to
as \texttt{d3.Selection}. The \texttt{data} argument was to take the full JSON \payload returned by
the parser. The \texttt{constructor} would then turn the \texttt{data} \texttt{Object} into a
\texttt{Map} object for more reliable access. To get more fine-grained access to the \svg document,
the \texttt{constructor} also populated extra fields with selections of the whole shield element
and the \texttt{<g id=``charge\_layers''>} element, defined within the \texttt{svg} selection. The
\texttt{chargesLayer} field is needed for telling the renderer classes where to draw their shapes.
Depending on the contents of the \payload, the \texttt{constructor} then instantiated new
\texttt{ChargeRenderer} objects for all the \texttt{\charges}, or instantiated 4 new
\texttt{Quarter} objects for all the \quarters.

\texttt{clearShield} was extracted from \texttt{index.ts} into \texttt{Blazon}, but with changes to
account for many \texttt{<path>} nodes as well as \texttt{<clipPath>} elements. The method now
iterated over child elements of the \texttt{<g id=``charge\_layers''>}, which were \texttt{<g>},
\texttt{<path>} and \texttt{<rect>} nodes, and \texttt{<clipPaths>} with an \texttt{id} beginning
with \texttt{``quarterly\_''}, deleting them in the process. It also stripped the \texttt{shield}
node of its CSS class.

\subsubsection{\texttt{Quarter} (Figure~\ref{fig:QuarterUML})}%
\label{ssub:quarter}

The new \texttt{Quarter} class relied on the order of the \texttt{quarters} array in the payload to
determine which quarter it was representing; with numbering starting at 0 in the top left quarter,
going left to right, ending at 3 in the bottom right. The \texttt{constructor} required the array
index as its first argument, with \texttt{field: ETincture}, \texttt{charges: ICharge[]},
\texttt{svg} and \texttt{chargesLayer} as its other arguments. With these parameters, the
\texttt{constructor} selected the quarter specified by the \texttt{index}, instantiated
\texttt{ChargeRenderer} objects --- borrowing a method from \texttt{Blazon} --- and a
\texttt{QuarterRenderer} object. The \texttt{draw()} method of \texttt{Quarter} first called the
\texttt{draw()} method on the \texttt{QuarterRenderer} object to draw the quarter \texttt{path}
onto the shield, then called \texttt{addClipPathDefinition(svg)} (see
Section~\ref{ssub:charge_renderer}) to add a new clip path to the \svg document. Having used
\texttt{QuarterRenderer} to draw a new quarter \texttt{path}, it selected it, assigned it to a
locally scoped constant, \texttt{quarterLayer}, and iterated over the \texttt{charges}. Before
calling the \texttt{draw} method on the \texttt{charge} object, it called the
\texttt{updateChargesLayer} on the \texttt{charge} object, passing it the \texttt{quarterLayer} to
instruct the \charge to render in the \quarter.

\subsubsection{\texttt{ChargeRenderer} (Figure~\ref{fig:charge_renderer_hierarchy})}%
\label{ssub:charge_renderer}

The \texttt{constructor} for \texttt{ChargeRenderer} expected a \texttt{chargesLayer}, a
\texttt{tincture: ETincture}, a \texttt{charge: ECharge | EQuarter} and a \texttt{sinister:
boolean} argument. The \texttt{tincture} argument was used by both the \texttt{ChargeRenderer}
itself and the \texttt{QuarterRenderer}, for the colour of the \charge and \gls{field}
respectively. The union type signature of \texttt{charge} allowed it to draw both \charges and
\quarters, as appropriate. For the drawing logic itself, I extracted the \texttt{drawCharge}
function from the \texttt{index.ts} file into the \texttt{ChargeRenderer} \texttt{draw()} method.
However, in trying to keep extensibility from \texttt{ChargeRenderer} to \texttt{QuarterRenderer},
the logic was then extracted from the \texttt{draw} method into its own protected
\texttt{drawCharge(}\texttt{currentCharge: d3.Selection,} \texttt{chargeLayer?: d3.Selection)}
method. This freed up the \texttt{draw} method to handle setting up the \svg document for the
\quarters and \charges to be drawn into, as well as applying clip paths.

Whilst implementing \charge rendering within \quarters, I found that I needed to change the
\texttt{chargesLayer} that \charges were being drawn in to. I decided to add a new method to update
the \texttt{chargesLayer} after \texttt{ChargeRenderer} instantiation.

The \texttt{getRandomInt} method was also extracted from the \texttt{index.ts} file
into \texttt{ChargeRenderer} for \texttt{id} generation.

\subsubsection{\texttt{QuarterRenderer} (Figure~\ref{fig:charge_renderer_hierarchy})}%
\label{ssub:quarter_renderer}

\texttt{QuarterRenderer} extended \texttt{ChargeRenderer}, with few changes to what it inherited.
In the \texttt{constructor}, rather than generating a random \texttt{id} for the quarter, it used
the \texttt{charge: ECharge | EQuarter} parameter (that would be a member of \texttt{EQuarter} in
this case). A new method \texttt{addClipPathDefinition(svg)} was also added. In \svg documents,
clip paths are defined using \texttt{<clipPath>} elements inside a \texttt{<defs>} element at the
top of the document. The \texttt{addClipPathDefinition} method selected the \texttt{<defs>} element
and appended a \texttt{<clipPath>} element, before updating the \texttt{chargesLayer} and calling
the inherited \texttt{drawCharge} method.

\subsection{Second Design Evaluation}%
\label{sub:second_design_evaluation}

This design iteration had brought quarterly rendering a lot closer to fruition, as seen in
Figure~\ref{fig:second_iteration_progress}, where  but I had difficulties maintaining clean code whilst
adding the functionality to apply quarter-specific transformations to scale and move \charges. To
add this, I had to redesign both how \texttt{ChargeShapes} and how both \texttt{Renderer} classes
worked.

This design iteration had brought quarterly rendering a lot closer to fruition, as seen in
Figure~\ref{fig:second_iteration_progress}, however proper scaling and translation of charges was
yet to be implemented. The project aims were yet to be realised. In prototyping to add the
functionality to apply quarter-specific transformations, I had difficulties maintaining clean,
extensible code. To both satisfy the project aims and the code standards, I had to redesign how
\texttt{ChargeShapes} and both \texttt{Renderer} classes worked.

\begin{figure}
  \centering
  \subfloat[The expected output for the blazon \textit{Quarterly: 1st and 4th: Gules, a bend
  Sable; 2nd and 3rd: Azure, a chief Or}.]{%
    \def\svgwidth{0.4\linewidth}
    \input{graphics/quarterly_example.pdf_tex}%
  }
  \qquad
  \subfloat[The second iteration output of this blazon.]{%
    \def\svgwidth{0.4\linewidth}%
    \input{graphics/second_iteration_quarterly.pdf_tex}
  }
  \caption{Second iteration progress.}%
  \label{fig:second_iteration_progress}
\end{figure}

%\subsection{SOLID Design Principles}%
%\label{sub:solid_design_principles}

%\todo{Refactor all of this, see photos, break principles into relevant sections.}

%SOLID object-oriented design is a collection of 5 design principles:

%\begin{itemize}
%  \item \textit{The Single Responsibility Principle} --- a class should have one, and only one,
%    reason to change;
%  \item \textit{The Open Closed Principle} --- you should be able to extend a class's behaviour,
%    without modifying it;
%  \item \textit{The Liskov Substitution Principle} --- derived classes must be substitutable for
%    their base classes;
%  \item \textit{The Interface Segregation Principle} --- make fine grained interfaces that are
%    client specific and
%  \item \textit{The Dependency Inversion Principle} --- depend on abstractions, not on concretions.
%\end{itemize}

\section{Refactoring Charge Rendering}%
\label{sec:refactoring_charge_rendering}

\subsection{Third Design Iteration}%
\label{sub:third_design_iteration}

\subsubsection{\texttt{Renderer} Hierarchy}%
\label{ssub:renderer_hierarchy}

This new design made use of the \textit{Liskov Substitution} and Open/Closed
principles~\autocite{martin2000design} (the Open/Closed principle being described in
Section~\ref{ssub:design_principles}). The Liskov Substitution principle dictates that derived
classes must be substitutable for their base class. This is well implemented with abstract classes
as they provide a reliable interface and force inheriting classes to implement the abstracted
methods defined within. As a reminder, the Open/Closed principle dictates that classes should be
closed to modification but open to extension.

Applying these principle, I redesigned the rendering structure to have common methods and
attributes belonging to both \texttt{ChargeRenderer} and \texttt{QuarterRenderer} in a new abstract
parent class, \texttt{Renderer}. This helped refine and enforce the previously specified API of
render classes having a main entry-point of \texttt{draw()}. Using an abstract class not only gave
assurances to classes that used its implementations that there would always be a \texttt{draw()}
function, but also gave child classes the shared functions that both required whilst allowing them
to specialise in their individual use cases. It also meant that previously ambiguous attribute
names could be named more specifically for the particular \texttt{Renderer} implementation. An
example of this being that where \texttt{ChargeRenderer} previously had an attribute of
\texttt{charge} with a union type of \texttt{ECharge | EQuarter}, the \texttt{charge} attribute
could now have a type of just \texttt{ECharge} and \texttt{QuarterRenderer} could have an attribute
of \texttt{quarter} with type \texttt{EQuarter}.

The planned execution for this design may be seen in Figure~\ref{fig:quarter_rendering}.

\subsubsection{\texttt{ChargeShapes} and \texttt{QuarterShapes} Abstraction}%
\label{ssub:chargeshape_and_quartershape_abstraction}

In a similar fashion to the \texttt{Renderer} redesign, I devised a new design for the attributes
and transforms for \texttt{ChargeShapes}. Rather than having all paths defined in a \texttt{Map} in
the \texttt{ChargeShapes}, a new abstract class, \texttt{AShape}, would be defined with a
\texttt{dimensions} \texttt{Map} and a \texttt{transforms(transform: string): string} method. Each
charge then extended \texttt{AShape}, implemented the two abstract properties and was imported by
the \texttt{ChargeShapes} class. The \texttt{ChargeShapes} was redesigned to have a boolean
function, \texttt{hasChargePath(charge: string)} to check if it knew the given charge.
\texttt{ChargeShapes} also had a static function, \texttt{chargeShapes(charge: ECharge)} which
returned an instance of \texttt{AShape}. This design makes use of the \textit{Dependency Inversion}
principle~\autocite{martin2000design}, which says that a class should depend on abstractions,
rather than concretions. In this case, when renderers call out to \texttt{ChargeShapes} for the
relevant \texttt{chargeShapes} object, they are returned an \texttt{AShape} object. This increases
reusability as well as extensibility of charge rendering as one just needs to create a new
\texttt{AShape} class, import it into the \texttt{ChargeShapes} class and add it to the relevant
methods.

Quarter shapes were also extracted into their own hierarchy, mirroring that of the \texttt{AShape}
hierarchy. \texttt{AQuarterShape} only exposed a \texttt{dimensions} \texttt{Map} however.

%Dependency Inversion - \texttt{AShape}, \texttt{AQuarterShape}

\begin{figure}
  \centering
  \subfloat[Initial \gls{escutcheon} outline on the webpage.]{%
    \def\svgwidth{0.425\linewidth}%
    \input{graphics/quarter_renderer_process/stage0.pdf_tex}
  }
  \qquad
  \subfloat[Drawing on the quarter shape.]{%
    \def\svgwidth{0.425\linewidth}%
    \input{graphics/quarter_renderer_process/stage1.pdf_tex}
  }
  \\
  \subfloat[Applying the \texttt{field} CSS class.]{%
    \def\svgwidth{0.425\linewidth}%
    \input{graphics/quarter_renderer_process/stage2.pdf_tex}
  }
  \qquad
  \subfloat[Drawing on the charge shape.]{%
    \def\svgwidth{0.425\linewidth}%
    \input{graphics/quarter_renderer_process/stage3.pdf_tex}
  }
  \\
  \subfloat[Applying the \texttt{tincture} CSS class.]{%
    \def\svgwidth{0.425\linewidth}%
    \input{graphics/quarter_renderer_process/stage4.pdf_tex}
  }
  \qquad
  \subfloat[Applying the scaling transformation to the charge.]{%
    \def\svgwidth{0.425\linewidth}%
    \input{graphics/quarter_renderer_process/stage5.pdf_tex}
  }
  \\
  \subfloat[Applying the clip path to the charge.]{%
    \def\svgwidth{0.425\linewidth}%
    \input{graphics/quarter_renderer_process/stage6.pdf_tex}
  }
  \caption{The process of the \texttt{Quarter} class rendering a quarter with a bend in it.}%
  \label{fig:quarter_rendering}
\end{figure}

\subsection{Third Design Implementation}%
\label{sub:third_design_implementation}

The Figure references in the following headings refer to the UML diagrams in
Appendix~\ref{sec:third_design_iteration_diagrams}.

\subsubsection{\texttt{Renderer} (Figure~\ref{fig:RendererUML})}%
\label{ssub:renderer}

The new \texttt{Renderer} class contained 3 protected, inherited attributes, \texttt{chargeId:
string}, \texttt{tincture: ETincture} and \texttt{parentChargesLayer: d3.Selection}. Whilst the
class had 3 attributes, only \texttt{tincture} and \texttt{parentChargesLayer} were assigned in the
constructor; inherited classes were given responsibility of generating their own \texttt{id}s. As
mentioned in Section~\ref{sub:third_design_iteration}, \texttt{Renderer} defined an abstract method
of \texttt{draw(): void} to be implemented in inherited classes. The \texttt{updateChargesLayer}
method was extracted from \texttt{ChargeRenderer} into the parent class to be available for both
child classes to inherit and use. \texttt{getRandomInt} was similarly extracted.

\subsubsection{\texttt{ChargeRenderer} (Figure~\ref{fig:RendererUML})}%
\label{ssub:chargerenderer}

Now inheriting from \texttt{Renderer}, \texttt{ChargeRenderer} had just two attributes defined on
it: \texttt{charge: ECharge} and \texttt{sinister: boolean}. The \texttt{draw} method now had an
optional argument of \texttt{quarter: EQuarter} which would correspond to the \texttt{id} of the
\texttt{<clipPath>} that contained this \charge. As before, the method performed environment setup
in the \svg document ready for the new \charge, then called out to \texttt{drawCharge}, however
after performing all this, \texttt{draw} now also called \texttt{applyTransforms}.
\texttt{applyTransforms} took in the \texttt{currentCharge} element and an optional
\texttt{quarter} argument, built a \texttt{transform} string based on the parameters, fetched the
appropriate transformation from \texttt{ChargeShapes} and applied it to the \texttt{currentCharge}.

\subsubsection{\texttt{QuarterRenderer} (Figure~\ref{fig:RendererUML})}%
\label{ssub:quarterrenderer}

As with \texttt{ChargeRenderer}, \texttt{QuarterRenderer} had only one attribute now defined on it:
\texttt{quarter: EQuarter}. Whereas before, the \texttt{draw} method called out to
\texttt{ChargeRenderer} to draw a quarter onto the shield, it now called its own
\texttt{drawQuarter} method. \texttt{drawQuarter} had much the same functionality as
\texttt{drawCharge} previously held, but exclusively worked on \texttt{QuarterShapes}.
\texttt{addClipPathDefinition} remained unchanged.

\subsubsection{\texttt{AShape} and \texttt{AQuarterShape} (Figure~\ref{fig:AShapeUML})}%
\label{ssub:ashape_and_aquartershape}

As mentioned in Section~\ref{ssub:chargeshape_and_quartershape_abstraction}, \texttt{AShape}
defined a pair of abstractions: the \texttt{dimensions} \texttt{Map} and the \texttt{transforms}
method. The \texttt{dimensions} \texttt{Map} was a simple \texttt{Map<string, string>} containing
relevant entries previously defined in the \texttt{ChargeShapes} class. \texttt{transforms} took a
transform string as an argument, with implementations using a \texttt{switch} statement to match
against it to specify the necessary transformation.

\texttt{AQuarterShape}, not needing any transformations, simply defined an abstract
\texttt{dimensions} \texttt{Map} that paralleled that of \texttt{AShape}.

\subsubsection{\texttt{ChargeShapes} (Figure~\ref{fig:ChargeShapesUML}) and \texttt{QuarterShapes}}%
\label{ssub:chargeshapes_and_quartershapes}

Having extracted all the previous dimensions into their own classes, \texttt{ChargeShapes} now
provided a single point-of-access for all \charges. It had a private, static array containing the
names of each imported \charge. To check whether the \charge could be drawn, it exposed a boolean
function, \texttt{hasChargePath(charge: string)} that scanned over the array to check whether the
argument given to it was contained within. A new function \texttt{chargeShapes(charge: ECharge)}
was defined to replace the newly-removed \texttt{chargePaths} \texttt{Map} which returned a new
object for the relevant \texttt{AShape}. When the \texttt{ChargeRenderer} called this, it was then able to
chain methods together without concerns about which \texttt{AShape} it had been given.

Again, to mirror the new \texttt{ChargeShapes} implementation, \texttt{QuarterShapes} was also
defined with a sole \texttt{quarterShapes(quarter: EQuarter)} which returned a new object for the
relevant \texttt{AQuarterShape}. The implementations for both \texttt{QuarterShapes} and
\texttt{AQuarterShape} were unnecessarily verbose, but were done purposefully to be keep the code
open-ended and extensible for future functionality.

\subsection{Third Iteration Evaluation}%
\label{sub:third_iteration_evaluation}

The project aim that I was pursuing was now realised, with quarterly shields being reliably drawn.
Some examples as drawn by the web app can be seen in Figure~\ref{fig:quarterly_shields}.

\begin{figure*}[h]
  \subfloat[A shield with the escutcheons of the 4 patron saints of the United Kingdom.]{%
    \def\svgwidth{0.3\linewidth}%
    \input{graphics/britain_shield.pdf_tex}
  }
  \qquad
  \subfloat[\textit{Quarterly: 1st and 4th: Gules, a bend Sable; 2nd and 3rd: Azure, a chief
  Or}.]{%
    \def\svgwidth{0.3\linewidth}%
    \input{graphics/quarterly_example.pdf_tex}
  }
  \caption{Some quarterly examples drawn by the web app.}%
  \label{fig:quarterly_shields}
\end{figure*}

\chapter{Evaluation}%
\label{cha:evaluation}

\section{Automated Testing}%
\label{sec:automated_testing}

I had troubles with setting up integration tests to ensure that the web app was calling the
libraries with correct parameters. However, as mentioned in
Section~\ref{sec:external_dependencies}, I used jest~\autocite{jest} to write unit tests for some
modules, specifically to ensure implementations of AShape were returning the correct data depending
on their inputs.

\section{Comparison With Existing Solutions}%
\label{sec:comparison_with_existing_solutions}

The other solution to this problem, as mentioned in Section~\ref{sec:related_works}, was
pyBlazon~\autocite{pyblazon} which does not support quarterly shields, so only basic
\glspl{escutcheon} may be compared. A comparison of both pyBlazon and this project rendering the
blazon for the Scrope \gls{escutcheon} can be seen in figure~\ref{fig:pyblazon_scrope_comparison}.
As noted in Section~\ref{sec:heraldry}, there are no shades specified for a tincture, hence the
difference in colours of pyBlazon's bend and this project's.

\todo{Add more comparisons.}

\begin{figure}
  \centering
  \subfloat[pyBlazon]{%
    %\def\svgwidth{0.425\linewidth}%
    %\input{graphics/pyblaze_scrope.pdf_tex}
    \includegraphics[width=0.45\linewidth]{pyblaze_scrope.png}
  }
  \qquad
  \subfloat[This project]{%
    \def\svgwidth{0.425\linewidth}%
    \input{graphics/scrope.pdf_tex}
  }
  \caption{A comparison of the Scrope \gls{escutcheon} (\textit{Azure, a bend Or}) being rendered
  by both this project and pyBlazon.}%
  \label{fig:pyblazon_scrope_comparison}
\end{figure}

\section{Shortcomings}%
\label{sec:shortcomings}

Due to time constraints, we were not able to integrate the front- and back-ends, with the server
configured to return mocked out responses for well-formed blazons.

In some quarters, the charges didn't fully fit, resulting in off-centred shapes to hide the visible
field. An example of this can be seen in Figure~\ref{fig:quarterly_shields}, where the bottom right
saltire is not aligned with the saltire above it. Similarly, the saltire in the top right corner
should sit much higher within the quarter. This was due to drawing the saltire to match the
dimensions of the shield outline, but not accounting for the differing dimensions of redrawing this
within a quarter. A fix for this particular issue would be to draw the shapes much larger than
their intended bounds and rely on \svg clip paths to keep the shape within the outline.

The project is currently only designed to accept correctly-formed data and does not handle
ill-formed data properly, short of throwing an exception. Ideally, it should be able to reject and
recover from ill-formed responses from the server, either through checking the validity of the data
before working on it or through ignoring data that cannot be processed. More feedback to the user
should also be given, should they enter a blazon that is unable to be parsed.

\chapter{Conclusion}%
\label{cha:conclusion}

\section{Overview}%
\label{sec:overview}

\todo{Write conclusion}

\section{Further Work}%
\label{sec:further_work}

Initial further work would be to make the fixes I have outlined in Section~\ref{sec:shortcomings}.

An immediate piece of further work would be to add more shapes to \texttt{ChargeShapes}, such that
the web app would be able to draw more \blazons. Given the current implementation, there is
potential to implement \glspl{variation}-based \blazons through use of \svg clip paths. Similarly,
adding support for drawing of \charges such as lions and fleur-de-lis might wished to be added.

\backmatter%

\printbibliography[heading=bibintoc]

\printglossary%

% vim: textwidth=99 spell spelllang=en_gb:
\graphicspath{{graphics/}} % set of paths to search for images
\DeclareGraphicsExtensions{.pdf,.png}

\appendix
\setboolean{@mainmatter}{true}

\chapter{Interfaces and Enums}%
\label{cha:interfaces_and_enums}

\begin{verbatim}
enum ETincture {
  /** For specifying Quarters */
  Quarterly = "quarterly",
  /** Gold/yellow */
  Or = "or",
  /** White */
  Argent = "argent",
  /** Blue */
  Azure = "azure",
  /** Red */
  Gules = "gules",
  /** Purple */
  Purpure = "purpure",
  /** Black */
  Sable = "sable",
  /** Green */
  Vert = "vert",
}

enum ECharge {
  Bend = "bend",
  Cross = "cross",
  Chief = "chief",
  Saltire = "saltire",
}

enum EQuarter {
  TL = "quarterly_tl",
  TR = "quarterly_tr",
  BL = "quarterly_bl",
  BR = "quarterly_br",
}

interface ICharge {
  charge: ECharge;
  sinister?: boolean;
  tincture?: ETincture;
}

interface IBlazon {
  field: ETincture;
  charges: ICharge[];
}
\end{verbatim}

A question mark on a field in an \texttt{interface} denotes it as optional.

\chapter{UML Diagrams}%
\label{cha:uml_diagrams}

In these UML diagrams, \texttt{d3.Selection} is a data type defined by D3.js~\autocite{d3js}. It
contains a reference to an HTML element for use in \dom{} manipulation.

\section{Second Design Iteration Diagrams}%
\label{sec:second_design_iteration_diagrams}

\begin{figure*}[h]
  %\centering
  \includegraphics[width=0.5\linewidth]{BlazonUML}
  \caption{\texttt{Blazon} UML.}%
  \label{fig:BlazonUML}
\end{figure*}

\begin{figure*}[h]
  %\centering
  \includegraphics[width=0.5\linewidth]{QuarterUML}
  \caption{\texttt{Quarter} UML.}%
  \label{fig:QuarterUML}
\end{figure*}

\begin{figure*}[h]
  %\centering
  \includegraphics[width=0.5\linewidth]{ChargeRendererUML}%
  \caption{\texttt{ChargeRenderer} hierarchy and methods.}%
  \label{fig:charge_renderer_hierarchy}
\end{figure*}

\pagebreak%

\section{Third Design Iteration Diagrams}%
\label{sec:third_design_iteration_diagrams}

\begin{figure*}[h]
  %\centering
  \includegraphics[width=0.8\linewidth]{RendererUML}
  \caption{\texttt{Renderer} hierarchy and methods.}%
  \label{fig:RendererUML}
\end{figure*}

\begin{figure*}[h]
  %\centering
  \includegraphics[width=0.5\linewidth]{ChargeShapesUML}
  \caption{\texttt{ChargeShapes} UML.}%
  \label{fig:ChargeShapesUML}
\end{figure*}

\begin{figure*}[h]
  %\centering
  \includegraphics[width=0.8\linewidth]{AShapeUML}
  \caption{\texttt{AShape} hierarchy and methods.}%
  \label{fig:AShapeUML}
\end{figure*}

\setboolean{@mainmatter}{false}


\end{document}
