% vim: textwidth=99 spell spelllang=en_gb:
\makeglossaries

\newglossaryentry{blazon}{%
  name={blazon},
  description={The text describing how an escutcheon is to be drawn}
}

\newglossaryentry{escutcheon}{%
  name={escutcheon},
  description={The shield in the coat of arms}
}

\newglossaryentry{quarter}{%
  name={quarter},
  description={A quarter within a divided escutcheon. Examples of quarterly escutcheons can be seen
  in Figure~\ref{fig:complex_shields}}
}

\newglossaryentry{field}{%
  name={field},
  description={The background colour of the escutcheon}
}

\newglossaryentry{ordinary}{%
  name={ordinary},
  description={Geometric shapes on the escutcheon},
  plural={ordinaries}
}

\newglossaryentry{charge}{%
  name={charge},
  description={Small emblems, such as fleur-de-lis and lions}
}

\newglossaryentry{variation}{%
  name={variation},
  description={Variations describe how the field or charge is patterned. Variations can indicate
  patterns such as chequered or coloured lines}
}

\newglossaryentry{tincture}{%
  name={tincture},
  description={The colours and patterns for charges, ordinaries and fields}
}

\newglossaryentry{minification}{%
  name={minification},
  description={Minification of JavaScript assets involves stripping out all
  unnecessary whitespace and tokens}
}

\newglossaryentry{uglification}{%
  name={uglification},
  description={Uglification transforms the JavaScript code by renaming all
  variables and functions into short, obfuscated names to reduce the footprint
  of the assets}
}

\newglossaryentry{transpile}{%
  name={transpile},
  description={Transpilation involves compiling from language to another, in this case compiling
  from TypeScript to JavaScript}
}

\newglossaryentry{dom}{%
  name={DOM},
  description={Document Object Model. See Section~\ref{sec:external_dependencies} for a description
  of the DOM},
  first={Document Object Model (DOM)}
}

\newglossaryentry{payload}{%
  name={payload},
  description={The JSON payload given to the web app from the parser. An example can be seen in
  Figure~\ref{fig:expected_output}}
}

\newacronym{json}{JSON}{JavaScript Object Notation}
\newacronym{nlp}{NLP}{Natural Language Processing}
\newacronym{nltk}{NLTK}{Natural Language Tool Kit}
\newacronym{cfg}{CFG}{Context-Free Grammar}
\newacronym{svg}{SVG}{Scalable Vector Graphics}
\newacronym{ci}{CI}{Continuous Integration}
